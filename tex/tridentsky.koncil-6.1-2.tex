%%%! luaLaTeX

\input tridentsky.koncil.macros

\title Tridentský koncil (6)~— část 1–2

\subtitle 6.~ZASEDÁNÍ slavené 13.~ledna 1547

\decret Dekret o~ospravedlnění


\partitle
Úvod

Protože se v~současnosti, nikoli bez újmy mnohých duší a~velké škody na jednotě církve,
rozšířilo jakési mylné učení o~ospravedlnění, zamýšlí posvátný ekumenický a~všeobecný
Tridentský koncil ke chvále a~slávě všemohoucího Boha, pro pokoj církve a~spásu duší
vyložit všem věřícím v~Krista pravé a~zdravé učení o~onom ospravedlnění, které „slunce
spravedlnosti“ (\bible{Mal 4,~2}) Ježíš Kristus, „původce a~dovršitel naší víry“ (\bible{Žid 12,~2}),
učil, apoštolové předali a~katolická církev z~popudu Ducha svatého neustále
zachovávala. Přísně se zapovídá věřit, hlásat nebo učit cokoli jiného, než je ustanoveno
a~vyhlášeno tímto dekretem.

\chaptitle
Kapitola 1:
Nedostatečnost přirozenosti a~zákona pro ospravedlnění lidí

Posvátný koncil v~prvé řadě prohlašuje, že k~řádnému a~zřetelnému pochopení učení
o~ospravedlnění je třeba, aby každý uznal a~vyznal, že když všichni lidé ztratili
v~Adamově provinění nevinnost (\bible{Řím 5,~12}; \bible{1~Kor 15,~22}), „stali se nečistými“ (\bible{Iz
64,~6}), a~— jak pravil Apoštol~— „svou přirozeností se stali syny hněvu“ (\bible{Ef 2,~3}).
Jak koncil vyložil v~dekretu o~dědičném hříchu, byli do té míry „služebníky hříchu“
(\bible{Řím 6,~20}) a~pod nadvládou ďábla a~smrti, že se ani pohanské národy silou lidské
přirozenosti, ani Židé literou samotného Mojžíšova Zákona od něj nemohli osvobodit
ani pozvednout. Jejich svobodná vůle nevyhasla, avšak zmenšily se a~ochably její
síly.

\chaptitle
Kapitola 2:
Boží úradek a~tajemství Kristova příchodu

Když nadešla ona blažená plnost času (\bible{Ef 1,~10; Gal 4,~4}), nebeský Otec, „Otec milosrdenství
a~Bůh veškeré útěchy“ (\bible{2~Kor 1,~3}), poslal lidem Ježíše Krista, svého Syna, ohlášeného
a~přislíbeného mnoha svatým Otcům před Zákonem i~v~čase Zákona (\bible{Gn 49,~10.18}), aby
vykoupil Židy, kteří byli pod Zákonem, a~aby „pohanské národy, které neusilovaly
o~spravedlnost, dosáhly spravedlnosti“ (\bible{Řím 9,~30}), a~tak aby všichni „byli přijati
za syny“ (\bible{Gal 4,~5}). „Pro naše hříchy ho Bůh ustanovil smírcem skrze víru v~jeho
krvi“ (\bible{Řím 3,~25}), a~to „nejen pro naše hříchy, ale i~pro hříchy celého světa“
(\bible{1~Jan 2,~2}).

\chaptitle
Kapitola 3:
Ti, kdo jsou v~Kristu ospravedlněni

Skutečně, i~když „zemřel za všechny“ (\bible{2 Kor 5,~15}), nepřijímají jeho dobrodiní všichni,
ale pouze ti, kteří mají účast na zásluze jeho utrpení. Vždyť lidé by se skutečně
nerodili spravedliví, kdyby se nerodili jako potomstvo Adamovo. Na základě tohoto
původu na sebe při početí přitahují nespravedlnost. A~tak kdyby se znovu nenarodili
v~Kristu, nebyli by nikdy ospravedlněni. S~tímto znovuzrozením je jim pro zásluhu
Kristova utrpení udělována milost, která je činí spravedlivými. Apoštol nás povzbuzuje,
abychom za toto dobrodiní bez ustání vzdávali díky Otci, „který nás uschopnil k~účasti
svatých ve světle“ (\bible{Kol 1,~12}), vytrhl nás z~moci temnosti a~přenesl do království
svého milovaného Syna, v~němž máme vykoupení a~odpuštění hříchů (\bible{Kol 1,~13n}).

\chaptitle
Kapitola 4:
Předběžný popis ospravedlnění hříšníka a~jeho údělu ve stavu milosti

Těmito slovy je naznačeno ospravedlnění hříšníka, což je přenesení ze stavu, do kterého
se člověk rodí jako syn prvního Adama, do stavu milosti „adoptivních synů“ (\bible{Řím 8,~15})
Božích skrze druhého Adama Ježíše Krista, našeho Spasitele. Toto přenesení není
od vyhlášení Evangelia možné bez koupele znovuzrození nebo bez touhy po ní, jak je
psáno: „Jestliže se nenarodí někdo z~vody a~z~Ducha svatého, nemůže vejít do Božího
království“ (\bible{Jan 3,~5}).

\chaptitle
Kapitola 5:
Nezbytnost přípravy na ospravedlnění v~dospělosti a~odkud ospravedlnění pochází

Dále koncil prohlašuje, že začátek tohoto ospravedlnění musí u~dospělých pocházet
z~předcházející Boží milosti skrze Krista Ježíše, to je z~jeho volání, kterým jsou
lidé povoláni bez jakýchkoli vlastních zásluh. A~tak ti, kteří byli pro hříchy odvráceni
od Boha, jsou skrze jeho podněcující a~pomáhající milost disponováni, aby se obrátili
ke svému vlastnímu ospravedlnění tím, že svobodně souhlasí a~spolupracují s~touto
milostí. Když se Bůh skrze osvícení Ducha svatého dotýká srdce člověka, nezůstává
člověk zcela nečinný, protože přijímá vnuknutí, byť ho může i~odmítnout. Zároveň
se ale bez milosti Boží nemůže před Bohem svobodnou vůlí pohnout ke spravedlnosti.
Když se tedy ve svatých písmech říká: „Obraťte se ke mně, a~já se obrátím k~vám“
(\bible{Zach 1,~3}), je nám připomínána naše svoboda. Když odpovídáme: „Obrať nás, Pane,
k~sobě, a~budeme obráceni“ (\bible{Pláč 5,~21}), vyznáváme, že Boží milost nás předchází.

\chaptitle
Kapitola 6:
Způsob přípravy

Lidé jsou disponováni k~samotné spravedlnosti, když podníceni a~podpořeni božskou
milostí přijímají víru „ze slyšení“ (\bible{Řím 10,~17}), svobodně směřují k~Bohu a~věří,
že je pravda to, co je božsky zjeveno a~přislíbeno, zvláště to, že Bůh ospravedlňuje
hříšníka milostí, „vykoupením v~Kristu Ježíši“ (\bible{Řím 3,~24}). Když poznávají, že jsou
hříšníci, obracejí se od strachu z~božské spravedlnosti, jíž jsou prospěšně otřeseni,
k~rozvažování o~Božím milosrdenství, pozvedají se v~naději s~důvěrou, že jim bude
Bůh kvůli Kristu nakloněn, a~začínají ho milovat jako pramen vší spravedlnosti.
Proto se staví vůči hříchům s~jistým odporem a~hnusem, což se projevuje pokáním,
které je třeba konat před křtem (\bible{Sk 2,~38}), teprve poté se rozhodují přijmout křest,
začít nový život a~zachovávat božská přikázání.

O~této přípravě je psáno: „Ten, kdo přichází k~Bohu, musí věřit, že je, a~že odměňuje
ty, kteří ho hledají“ (\bible{Žid 11,~6}). Dále: „Důvěřuj synu, jsou ti odpuštěny hříchy“
(\bible{Mt 9,~2}; \bible{Mk 2,~5}). A~dále: „Bázeň před Bohem zahání hřích“ (\bible{Sir 1,~27}). A~dále: „Čiňte
pokání a~dejte se všichni pokřtít ve Jménu Ježíše Krista na odpuštění svých hříchů,
a~dostanete dar Ducha svatého“ (\bible{Sk 2,~38}). Dále: „Jděte, učte všechny národy, křtěte
je ve jménu Otce i~Syna i~Ducha svatého a~učte je zachovávat vše, co jsem vám přikázal“
(\bible{Mt 28,~19}). A~konečně: „Připravte svá srdce Pánu“ (\bible{1~Král 7,~3}).

\chaptitle
Kapitola 7:
Co je ospravedlnění hříšníka a~jaké jsou jeho příčiny

Po tomto disponování se neboli přípravě následuje samotné ospravedlnění, které není
pouhým odpuštěním hříchů, ale i~posvěcením a~obnovou lidského nitra svobodným přijetím
milosti a~darů. Tak se z~nespravedlivého člověka stává člověk ospravedlněný a~z~nepřítele
přítel, aby byl „podle naděje dědicem života věčného“ (\bible{Tit 3,~7}).

Příčiny tohoto ospravedlnění jsou tyto: příčina účelová: sláva Boží a~Kristova a~pak
život věčný; příčina účinná: milosrdný Bůh, který zdarma očišťuje a~posvěcuje
(\bible{1~Kor 6,~11}) tak, že označuje a~pomazává přislíbeným Duchem svatým, který je závdavkem
našeho dědictví (\bible{Ef 1,~13n}); příčina záslužná je nejmilovanější jednorozený Syn,
náš Pán Ježíš Kristus, který nám~— „když jsme byli nepřáteli“ (\bible{Řím 5,~10})~— „pro
svou nesmírnou lásku, kterou nás miloval“ (\bible{Ef 2,~4}), zasloužil ospravedlnění svým
nejsvětějším utrpením na dřevě kříže a~dal za nás Bohu Otci zadostiučinění. Příčinou
nástrojovou je svátost křtu, což je svátost víry, bez které nedosáhne vůbec nikdo
ospravedlnění. A~konečně, jedinou formální příčinou je spravedlnost Boží. Ne ta,
kterou je on sám spravedlivý, ale ta, kterou nás činí spravedlivými. Touto milostí,
kterou nám daroval, jsme totiž vnitřně obnovováni na duchu a~jsme nejen pokládáni
za spravedlivé, ale opravdu se spravedlivými nazýváme a~jsme. Každý z~nás přijímá
spravedlnost, kterou „Duch svatý rozděluje každému, jak chce“ (\bible{1~Kor 12,~11}), úměrně
dispozici a~spolupráci dotyčného člověka.

A~tak nikdo nemůže být spravedlivý, pokud nemá účast na zásluhách utrpení našeho
Pána Ježíše Krista. Tato účast spočívá u~ospravedlněného hříšníka v~tom, že se pro
zásluhu nejsvětějšího utrpení skrze Ducha svatého rozlévá v~srdcích ospravedlněných
(\bible{Řím 5,~5}) láska Boží, která v~nich zůstává. Z~toho plyne, že v~samotném ospravedlnění,
které se vlévá zároveň s~odpuštěním hříchů, člověk přijímá skrze Ježíše Krista, do
něhož je vštípen, víru, naději a~lásku.

Jestliže ale k~víře nepřistupuje naděje a~láska, pak víra ani dokonale nesjednocuje
s~Kristem, ani nevytváří živý úd jeho těla. Z~tohoto důvodu se velmi správně říká,
že „víra bez skutků je mrtvá“ (\bible{Jak 2,~17nn}) a~nečinná. „V Kristu Ježíši nezáleží
na tom, je\=li někdo obřezán či ne; rozhodující je víra, která se uplatňuje láskou“
(\bible{Gal 5,~6; 6,~15}). Tuto víru žádají podle apoštolské tradice od církve katechumeni
před svátostí křtu, když prosí o~víru, která dává život věčný. Ten však víra bez
naděje a~lásky dát nemůže. Proto hned slyší Kristovo slovo: „Chceš\=li vejít do života,
zachovávej přikázání“ (\bible{Mt 19,~17}). Když tedy ti, kdo jsou křtem znovuzrození, přijímají
pravou a~křesťanskou spravedlnost, mají ji jako nejlepší roucho (\bible{Lk 15 ,22}), které
jim bylo dáno skrze Ježíše Krista místo toho, které Adam svou neposlušností sobě
i~nám ztratil, uchovat čistou a~neposkvrněnou, aby ji přinesli před soud našeho Pána
Ježíše Krista a~měli život věčný.

\chaptitle
Kapitola 8:
Jak rozumět tomu, že hříšník je ospravedlněn vírou a~zdarma

Když Apoštol řekl, že člověk je ospravedlněn vírou a~„zdarma“ (\bible{Řím 3,~22–24}), je
třeba těmto slovům rozumět tak, jak je katolická církev vždy chápala a~vykládala.
Říká se, že jsme ospravedlněni vírou, protože „víra je počátkem lidské spásy“, základem
a~kořenem veškerého ospravedlnění. „Bez ní se nelze líbit Bohu“ (\bible{Žid 11,~6}), ani
dosáhnout společenství jeho synů. Říká se, že jsme ospravedlněni zdarma, protože
ničím z~toho, co ospravedlnění předchází, ani vírou, ani skutky, se samotná milost
ospravedlnění nezasluhuje. „Jestliže je totiž ospravedlnění z~milosti, pak není
ze skutků“, jak pravil Apoštol, „jinak by milost nebyla milostí“ (\bible{Řím 11,~6}).

\chaptitle
Kapitola 9:
Proti marné důvěře heretiků

Přestože je nezbytné věřit, že se hříchy neodpouští, ani že se nikdy neodpouštěly
bez božské milosti dané zdarma kvůli Kristu, je třeba říci, že nikomu, kdo se chlubí
důvěrou a~jistotou ohledně odpuštění hříchů a~utěšuje se jenom tím, hříchy odpuštěny
nejsou ani nebyly. Ačkoli, u~heretiků a~schismatiků by mohla být, a~v~těchto časech
dokonce je, s~velikou horlivostí vůči katolické církvi hlásána i~tato prázdná a~veškeré
zbožnosti zbavená důvěra.

A~nelze ani tvrdit, že ti, kteří jsou skutečně ospravedlněni, musí být bez jakékoli
pochybnosti přesvědčeni, že jsou ospravedlnění, ani že nikdo není rozhřešen a~ospravedlněn
kromě toho, který s~jistotou věří, že je rozhřešen a~ospravedlněn, a~že rozhřešení
a~ospravedlnění je působeno pouze vírou, jako by pak ten, kdo takto nevěří, pochyboval
o~Božích přislíbeních a~účinnosti Kristovy smrti a~vzkříšení. Vždyť nikdo zbožný
nesmí pochybovat o~Božím milosrdenství, o~Kristově zásluze ani o~účinnosti svátostí.
Když se ale kdokoli dívá na sebe a~na svou vlastní slabost a~nedostatečnou dispozici,
může se obávat a~strachovat o~svou milost. Nikdo přece není schopen vědět s~jistotou
víry, která nemůže podléhat omylu, že dosáhl milosti Boží.

\chaptitle
Kapitola 10:
Růst přijatého ospravedlnění

Ti, kdo jsou ospravedlněni, se stali „přáteli Božími“ a~„Boží rodinou“ (\bible{Jan 15,~15;
Ef 2,~19}), „kráčejí od ctnosti ke ctnosti“ (\bible{Ž~84,~8}) a~„den ze dne se obnovují“
(\bible{2~Kor 4,~16}), jak praví Apoštol, „když umrtvují údy svého těla“ (\bible{Kol 3,~5}) k~tomu,
aby se posvěcovali zachováváním Božích i~církevních přikázání. V~této spravedlnosti
přijaté skrze Kristovu milost, rostou „spolupůsobením víry a~dobrých skutků“ (srov.
\bible{Jak 2,~22}) a~míra jejich ospravedlnění se zvětšuje, jak je psáno: „Kdo je spravedlivý,
ať zůstane spravedlivý“ (\bible{Zj 22,~11}). A~opět: „Nečekej s~ospravedlněním až do smrti“
(\bible{Sir 18,~22}). A~znovu: „Vidíte, že ze skutků je člověk ospravedlněný, a~ne pouze
z~víry“ (\bible{Jak 2,~24}). O~tento vzrůst spravedlnosti prosí svatá církev, když se modlí:
„Rozhojni v~nás, Pane, víru, naději a~lásku“ (modlitba 13. neděle po Svatém Duchu).

\chaptitle
Kapitola 11:
Zachovávání přikázání, jeho nezbytnost a~možnost

Nikdo se však, ať už je jakkoli ospravedlněný, nesmí považovat za osvobozeného od
zachovávání přikázání. Nikdo se nesmí řídit opovážlivostmi, které Otcové zakázali
a~postihli anathematem, že je totiž pro ospravedlněného nemožné zachovávat Boží přikázání.
„Vždyť Bůh nepřikazuje nic nemožného. Naopak, přikazováním napomíná k~tomu, abys
činil, co můžeš, abys žádal o~to, co sám činit nemůžeš“, a~pomáhá tomu, abys činit
mohl. „Jeho přikázání nejsou těžká“ (\bible{1~Jan 5,~3}). „Jeho jho je sladké a~břemeno lehké“
(srov. \bible{Mt 11,~30}). Ti totiž, kteří jsou  syny Božími, Krista milují, a~jak on sám
dosvědčuje, ti, kdo ho milují, zachovávají jeho slova (\bible{Jan 14,~23}). S~Boží pomocí
je to opravdu možné.

Ačkoli v~tomto smrtelném životě všichni, ať už jsou jakkoli svatí a~spravedliví,
upadají přinejmenším do lehkých, každodenních hříchů, kterým se říká hříchy všední,
nepřestávají být kvůli tomu spravedliví. Vždyť hlas spravedlivých: „Odpusť nám naše
viny“ (\bible{Mt 6,~12}) je pokorný a~pravdivý. Proto se sami ospravedlnění musí cítit o~to
víc zavázáni kráčet cestou spravedlnosti. Jakožto „osvobození od hříchu se stávají
služebníky Božími“ (srov. \bible{Řím 6,~22}) a~„žijíce střízlivě, spravedlivě a~zbožně“ (\bible{Tit
2,~12}), mohou prospívat skrze Ježíše Krista, skrze něhož mají přístup k~této milosti
(\bible{Řím 5,~2}). Vždyť Bůh neopouští svou milostí ty, kteří už jsou ospravedlněni, jestliže
ho předtím sami neopustili.

Proto si nikdo nesmí lichotit, že má pouze víru, a~myslet, že samotnou vírou je ustanoven
za dědice a~že dědictví dosáhne, i~kdyby spolu s~Kristem netrpěl, aby s~ním také
mohl být oslavený (srov. \bible{Řím 8,~17}). Vždyť i~sám Kristus, jak praví Apoštol, „ačkoli
to byl Syn Boží, naučil se poslušnosti z~utrpení, jímž prošel, tak dosáhl dokonalosti
a~všem, kteří ho poslouchají, se stal původcem věčné spásy“ (\bible{Žid 5,~8n}). Proto sám
Apoštol ospravedlněné napomíná, když říká: „Nevíte snad, že ti, kteří běží na cvičišti,
běží sice všichni, ale jen jeden dostane cenu? Běžte tak, abyste ji získali! Já tedy
běžím ne jako bez cíle; bojuji ne tak, jako bych dával rány do prázdna, ale trestám
své tělo a~podrobuji si ho, abych snad, když kážu jiným, sám nebyl zavržen“ (\bible{1~Kor
9,~24nn}). Rovněž první z~apoštolů Petr: „Snažte se upevňovat své povolání a~vyvolení.
Budete\=li to činit, nikdy nezhřešíte“ (\bible{2~Petr 1,~10}).

Proto je jisté, že nauce pravověrného náboženství protiřečí ti, kteří říkají, že
spravedlivý každým dobrým skutkem páchá aspoň všední hřích, nebo, což je ještě nepřijatelnější,
že si onen spravedlivý zasluhuje věčné tresty. Dále také ti, kteří tvrdí, že spravedliví
hřeší ve všech svých skutcích, pokud kvůli vyburcování ze své netečnosti a~povzbuzení
v~běhu na cvičišti sledují vedle prvotního cíle, kterým je oslava Boží, také věčnou
odměnu. Je totiž psáno: „Mé srdce se sklonilo, abych pro odplatu konal tvé spravedlivé
skutky“ (\bible{Ž~119,~112}), a~o~Mojžíšovi praví Apoštol: „Upíral svou mysl k~budoucí odplatě“
(\bible{Žid 11,~26}).

\chaptitle
Kapitola 12:
Je se třeba chránit opovážlivého předpokladu o~předurčení

Nikdo, dokud žije v~tomto stavu smrtelnosti, nesmí ohledně božského tajemství předurčení
smýšlet tak, aby tvrdil, že je naprosto jistě v~počtu předurčených, jako by byla
pravda, že ospravedlněný buď nemůže dále hřešit, nebo když už zhřeší, s~jistotou
předvídá své obrácení. Vždyť bez zvláštního zjevení není možné vědět, koho si Bůh
vyvolil.

\chaptitle
Kapitola 13:
Dar vytrvalosti

Podobně si nikdo nesmí s~absolutní jistotou slibovat něco ohledně daru vytrvalosti,
o~kterém je psáno: „Kdo vytrvá až do konce, bude spasen“ (\bible{Mt 10,~22; 24,~13}), a~který
ovšem nelze obdržet od nikoho jiného, než od toho, který „může toho, kdo stojí, podepřít“
(srov. \bible{Řím 14,~4}), aby stál vytrvale, a~toho, kdo padl, znovu pozvednout. Každý však
musí skládat a~uchovávat nejpevnější naději v~Boží pomoc. Jestliže totiž Bůh začal
dobré dílo, tak ho~— pokud samo lidé jeho milost nezanedbávají~— i~dokončí, neboť
působí chtění i~dokonání (\bible{Flp 2,~13}). A~proto ti, kdo si myslí, že stojí, ať si dají
pozor, aby nepadli (\bible{1~Kor 10,~12}), a~s~bázní a~chvěním ať uvádějí ve skutek své spasení
(\bible{Flp 2,~12}), v~práci, v~bdění, v~almužnách, v~modlitbách a~obětech, v~postech a~čistotě
(srov. \bible{2~Kor 6,~3nn}). Proto když vědí, že byli znovuzrozeni k~naději (srov. \bible{1~Petr
1,~3}) na slávu, a~ne ještě ke slávě samé, musí se obávat boje, který zbývá vybojovat
s~tělem, světem a~ďáblem, a~ve kterém se nemohou stát vítězi, pokud s~milostí Boží
neposlouchají Apoštola, který říká: „Jsme dlužní, ale ne tělu, abychom museli žít
podle těla. Žijete\=li podle těla, umíráte. Jestliže však Duchem umrtvíte skutky těla,
budete žít“ (\bible{Řím 8,~12n}).

\chaptitle
Kapitola 14:
O~těch, kteří upadli do hříchu, a~o~jejich obnově

Ti však, kdo hříchem od přijaté milosti ospravedlnění odpadli, mohou být znova ospravedlněni,
když podníceni Bohem znovuzískají ve svátosti pokání na základě Kristovy zásluhy
ztracenou milost. Tento způsob ospravedlnění je obnovou toho, který padl. Svatí Otcové
ho vhodně prohlašovali za „druhé záchranné prkno po ztroskotání ztracené milosti“.
Vždyť přece pro ty, kteří po křtu klesají do hříchů, ustanovil Kristus Ježíš svátost
pokání, když řekl: „Přijměte Ducha svatého. Komu hříchy odpustíte, tomu jsou odpuštěny,
komu je neodpustíte, tomu odpuštěny nejsou“ (\bible{Jan 20,~22–23}).

Proto je třeba učit, že pokání padlého křesťana je zcela jiné než pokání křestní.
Obsahuje nejen ustoupení od hříchů a~jejich odmítnutí neboli „zkroušené a~pokorné
srdce“ (\bible{Ž~51,~19}), ale i~jejich svátostné vyznání, přinejmenším předsevzetí vykonat
ho v~příslušný čas, a~kněžské rozhřešení. Dále pak zadostiučinění posty, almužnami,
modlitbami a~jinými zbožnými cvičeními duchovního života. Ne však kvůli věčným trestům,
které se svátostí nebo touhou po svátosti odpouštějí spolu s~vinou, ale kvůli trestu
časnému, který se podle učení svatých Písem neodpouští vždy úplně tak, jako je tomu
při křtu, těm, kteří nevděčností k~Boží milosti, kterou přijali, zarmoutili Ducha
svatého (srov. \bible{Ef 4,~30}) a~neobávali se poskvrnit chrám Boží (srov. \bible{1~Kor 3,~17}).
O~tomto pokání je psáno: „Rozpomeň se, odkud jsi klesl, navrať se a~jednej jako dřív“
(\bible{Zj 2,~5}). A~dále: „Zármutek podle Boží vůle působí pokání ke spáse“ (\bible{2~Kor 7,~10}).
A~dále: „Neste tedy ovoce, které ukazuje, že činíte pokání“ (\bible{Mt 3,~8}).

\chaptitle
Kapitola 15:
Jakýmkoli smrtelným hříchem se ztrácí milost, ale ne víra

Také proti vychytralému důvtipu některých lidí, kteří „sladkými řečmi a~chválami
svádějí srdce nevinných“ (\bible{Řím 16,~18}), je třeba zřetelně říci: přijatá milost ospravedlnění
se ztrácí nejen nevěrou, kterou je ztracena i~samotná víra, ale také každým jiným
smrtelným hříchem, při kterém se víra neztrácí. Tím je bráněno učení božského zákona,
které z~Božího království vylučuje nejen nevěřící, ale i~věřící, kteří jsou „smilníci,
cizoložníci, chlípníci, souložníci mužů, zloději, lakomci, opilci, utrhači, lupiči“
(srov. \bible{1~Kor 6,~9n}) a~všichni ostatní, kteří se dopouštějí smrtelných hříchů, jichž
se mohou s~pomocí Boží milosti zdržet, a~kvůli kterým se oddělují od Kristovy milosti.

\chaptitle
Kapitola 16:
O~ovoci ospravedlnění, tj. o~zásluze dobrých skutků a~o~povaze samotné zásluhy

Lidem takto ospravedlněným, ať už přijatou milost bez přestání zachovávají nebo
ji ztratí a~opět získají, jsou určena slova Apoštola: Buďte stále horlivější v~každém
dobrém díle; vždyť „víte, že vaše práce není v~Pánu marná“ (\bible{1~Kor 15,~58}). „Bůh
není nespravedlivý, a~proto nezapomene na vaše skutky a~lásku, kterou jste projevili
v~jeho jménu“ (\bible{Žid 6,~10}). A~dále: „Neztrácejte proto odvahu, neboť bude bohatě
odměněna“ (\bible{Žid 10,~35}). Před oči těch, kteří „až do samého konce“ (\bible{Mt 10,~22}) dobře
jednají a~doufají v~Boha, je proto třeba postavit věčný život jakožto milost Božích
synů milosrdně přislíbenou skrze Krista i~jako odměnu, která jim má být na základě
téhož Božího přislíbení věrně předána za jejich dobré skutky a~zásluhy. To je onen
věnec spravedlnosti, o~němž Apoštol říkal, že je pro něj po jeho boji a~běhu připraven,
a~že mu ho předá spravedlivý soudce; a~nejen jemu, ale také všem ostatním, kteří
milují jeho příchod (\bible{2~Tim 4,~7n}).

Tento Kristus Ježíš jako „hlava údům“ (\bible{Ef 4,~15}) a~jako vinný kmen ratolestem (\bible{Jan
15,~5}) ustavičně vlévá ospravedlněným sílu. Tato síla vždy předchází, doprovází a~následuje
jejich dobré skutky. Bez ní by tyto dobré skutky v~žádném případě nemohly být Bohu
milé a~záslužné. Je třeba věřit, že ospravedlněným nechybí nic více k~tomu, aby svými
skutky, které jsou vykonány v~Bohu, plně zadostiučinili božskému zákonu týkajícímu
se tohoto života, a~aby si ve svém čase, vytrvají\=li v~milosti (\bible{Zj 14,~13}), zasloužili
život věčný. Náš spasitel Kristus přece říká: Kdo se napije vody, kterou mu dám já,
nebude žíznit navěky, ale bude v~něm pramen vody, vyvěrající k~životu věčnému (\bible{Jan
4,~14}).

Tak nebude naše vlastní spravedlnost pokládána za něco vlastního, co vychází z~nás,
ani nebude ignorována nebo odmítána spravedlnost Boží (\bible{Řím 10,~3}). Této spravedlnosti
se říká naše, protože se nachází v~nás a~skrze ni jsme ospravedlňováni. Táž spravedlnost
je však Boží, protože je nám vlévána Bohem pro Kristovu zásluhu.

Ačkoli se dobré skutky ve svatých Písmech oceňují až tak vysoko, že Kristus přislibuje,
že kdo by dal číši studené vody jednomu z~jeho maličkých, ten nepřijde o~svou odměnu
(\bible{Mt 10,~42}), a~že Apoštol svědčí, že toto naše krátké a~lehké soužení působí přenesmírnou
váhu věčné slávy (\bible{2~Kor 4,~17}), nesmí se opomíjet ani to, že křesťan má být dalek
toho, aby důvěřoval v~sebe samého nebo v~sobě samém hledal svou chloubu namísto u~Pána
(srov. \bible{1~Kor 1,~31; 2~Kor 10,~17}). Jeho dobrota ke všem lidem je tak veliká, že si
přeje, aby jeho dary byly jejich zásluhami.

A~protože „všichni mnoho chybujeme“ (\bible{Jak 3,~2}), každý musí mít před očima jak milosrdenství
a~dobrotu, tak přísnost a~soud. Nikdo dále nemůže sám sebe soudit, i~kdyby si nebyl
vědom žádné viny. Život každého člověka má být totiž zkoumán a~souzen nikoli soudem
lidským, ale soudem Božím, který „vynese na světlo to, co je skryto ve tmě, a~zjeví
záměry srdcí; tehdy se člověku dostane chvály od Boha“ (\bible{1~Kor 4,~4n}), který, jak
je psáno, „odplatí každému podle jeho skutků“ (\bible{Řím 2,~6}).

K~tomuto katolickému učení o~ospravedlnění, bez jehož věrného a~pevného přijetí nemůže
být nikdo ospravedlněn, uznal posvátný koncil za vhodné připojit následující kánony,
aby všichni věděli nejen to, čeho se mají držet a~co mají následovat, ale také to,
čemu se musí vyhýbat a~od čeho musí utíkat.



\canonMainTitle Kánony o~ospravedlnění

\canon
Kánon 1: Kdyby někdo řekl, že se člověk může před Bohem ospravedlnit svými skutky
vykonanými silami lidské přirozenosti nebo podle nauky Zákona bez božské milosti
skrze Ježíše Krista, anathema sit.

\canon
Kánon 2: Kdyby někdo řekl, že božská milost byla skrze Ježíše Krista dána pouze k~tomu,
aby člověk snadněji mohl žít spravedlivě a~aby si mohl zasloužit věčný život, jako
by obojího mohl dosáhnout~— byť namáhavě a~s~obtížemi~— svobodnou vůlí bez milosti,
anathema sit.

\canon
Kánon 3: Kdyby někdo řekl, že bez předcházejícího vnuknutí Ducha svatého a~bez jeho
pomoci může člověk dostatečně věřit, doufat, milovat nebo se kát, aby mu byla udělena
milost ospravedlnění, anathema sit.

\canon
Kánon 4: Kdyby někdo řekl, že Bohem pohnutá a~vzbuzená svobodná vůle člověka nijak
nespolupracuje s~tím, že souhlasí s~Bohem, který povzbuzuje a~volá, takže se ona
lidská svobodná vůle nedisponuje a~nepřipravuje, ani že nemůže~— i~kdyby chtěla~—
odporovat, ale jako něco neživého nekoná vůbec nic a~chová se zcela pasivně, anathema
sit.

\canon
Kánon 5: Kdyby někdo řekl, že se svobodná vůle člověka po Adamově hříchu ztratila
a~vyhasla, nebo že je pouhým označením nebo přímo označením bezobsažným, nebo dokonce
výplodem vneseným satanem do církve, anathema sit.

\canon
Kánon 6: Kdyby někdo řekl, že není v~moci člověka, aby učinil své cesty zlými, ale
že jak zlé, tak dobré skutky způsobuje Bůh, a~to tak, že je nejen dopouští, ale také
v~pravém smyslu a~sám od sebe působí, až do té míry, že Jidášova zrada není jeho
dílem méně než povolání Pavla, anathema sit.

\canon
Kánon 7: Kdyby někdo řekl, že všechny skutky učiněné před ospravedlněním, ať už byly
vykonány z~jakéhokoli důvodu, jsou ve skutečnosti hříchy nebo že zasluhují Boží nenávist,
nebo že čím více člověk usiluje disponovat se k~milosti, tím více hřeší, anathema
sit.

\canon
Kánon 8: Kdyby někdo řekl, že strach před peklem, pro který se v~bolesti nad hříchy
utíkáme k~Božímu milosrdenství nebo pro který se hřešení zdržujeme, je hříchem nebo
že dělá hříšníky ještě horšími, anathema sit.

\canon
Kánon 9: Kdyby někdo řekl, že hříšníka ospravedlňuje pouhá víra, a~chápal to tak,
že se nevyžaduje nic jiného, čím by spolupracoval na dosažení milosti ospravedlnění,
a~že není vůbec potřeba připravit se a~disponovat činností vlastní vůle, anathema
sit.

\canon
Kánon 10: Kdyby někdo řekl, že lidé jsou ospravedlněni bez Kristovy spravedlnosti,
skrze kterou nám zasloužil ospravedlnění, nebo že skrze samu tuto spravedlnost jsou
formálně spravedliví, anathema sit.

\canon
Kánon 11: Kdyby někdo řekl, že lidé jsou ospravedlněni buď pouhým připsáním Kristovy
spravedlnosti nebo pouhým odpuštěním hříchů, bez milosti a~lásky, které se rozlévají
skrze Ducha svatého v~jejich srdcích a~pronikají je, nebo že také milost, kterou
jsme ospravedlněni, je pouze přízní Boží, anathema sit.

\canon
Kánon 12: Kdyby někdo řekl, že ospravedlňující víra není nic jiného než důvěra v~božské
milosrdenství, které kvůli Kristu odpouští hříchy, nebo že ona pouhá důvěra je to,
čím jsme ospravedlněni, anathema sit.

\canon
Kánon 13: Kdyby někdo řekl, že k~dosažení odpuštění hříchů je pro každého člověka
nezbytné, aby jistě a~bez jakéhokoli váhání způsobeného vlastní slabostí a~nedisponováním
věřil, že jsou mu odpuštěny hříchy, anathema sit.

\canon
Kánon 14: Kdyby někdo řekl, že člověk je od hříchů osvobozen a~ospravedlněn tím,
že s~jistotou věří ve své osvobození a~ospravedlnění, anebo že nikdo není opravdu
ospravedlněn, nevěří\=li, že je ospravedlněn, a~že pouze touto vírou dosahuje osvobození
od hříchů a~ospravedlnění, anathema sit.

\canon
Kánon 15: Kdyby někdo řekl, že znovuzrozený a~ospravedlněný člověk je z~víry zavázán
věřit, že je s~jistotou v~počtu předurčených, anathema sit.

\canon
Kánon 16: Kdyby někdo s~naprostou a~neomylnou jistotou řekl, že dosáhne onoho velkého
daru vytrvalosti až do konce, aniž by se to dověděl ze zvláštního zjevení, anathema
sit.

\canon
Kánon 17: Kdyby někdo řekl, že milost ospravedlnění se týká pouze těch, kteří jsou
předurčeni k~životu, a~že všichni ostatní povolaní jsou sice povoláni, ale milost
nedostávají, protože jsou božskou mocí předurčeni ke zlému, anathema sit.

\canon
Kánon 18: Kdyby někdo řekl, že zachovávat Boží přikázání je nemožné i~pro člověka
ospravedlněného a~stojícího pod milostí, anathema sit.

\canon
Kánon 19: Kdyby někdo řekl, že v~Evangeliu není přikázáno nic kromě víry, a~na ostatních
věcech že nezáleží, že nejsou ani přikázané, ani zakázané, ale svobodné, nebo že
se Desatero nevztahuje na křesťany, anathema sit.

\canon
Kánon 20: Kdyby někdo řekl, že ospravedlněný a~jakkoli dokonalý člověk není vázán
zachovávat Boží a~církevní přikázání, ale pouze věřit, jako by Evangelium nebylo
ničím jiným než absolutním příslibem věčného života bez podmínky zachování přikázání,
anathema sit.

\canon
Kánon 21: Kdyby někdo řekl, že Kristus Ježíš byl od Boha dán lidem jako vykupitel,
kterému mají věřit, a~ne také jako zákonodárce, kterého mají poslouchat, anathema
sit.

\canon
Kánon 22: Kdyby někdo řekl, že ospravedlněný může bez zvláštní pomoci Boží vytrvat
v~přijaté spravedlnosti, nebo že s~touto pomocí vytrvat nemůže, anathema sit.

\canon
Kánon 23: Kdyby někdo řekl, že jednou ospravedlněný člověk nemůže nadále hřešit ani
ztratit milost, a~tudíž ten, který upadá a~hřeší, nebyl nikdy opravdu ospravedlněn,
nebo naopak, že se může v~průběhu celého života vyhnout všem hříchům, i~lehkým, a~to
bez zvláštního Božího privilegia, jak to církev věří o~blahoslavené Panně, anathema
sit.

\canon
Kánon 24: Kdyby někdo řekl, že přijatá spravedlnost se před Bohem nezachovává ani
nerozhojňuje dobrými skutky, ale že samotné skutky jsou pouze plody a~znamení přijatého
ospravedlnění, ne však příčinou jeho rozhojnění, anathema sit.

\canon
Kánon 25: Kdyby někdo řekl, že spravedlivý přinejmenším lehce hřeší v~každém dobrém
skutku, nebo~— což je ještě méně omluvitelné~— že tak hřeší smrtelně, a~proto zasluhuje
věčný trest, a~že není zavržen jen proto, že mu Bůh ony skutky k~zavržení nepřičítá,
anathema sit.

\canon
Kánon 26: Kdyby někdo řekl, že spravedliví nemají pro své dobré skutky, které byly
učiněny v~Bohu, doufat a~očekávat od Boha věčnou odměnu skrze jeho milosrdenství
a~zásluhu Ježíše Krista, když v~dobrém jednání a~zachovávání božských přikázání vytrvají
až do konce, anathema sit.

\canon
Kánon 27: Kdyby někdo řekl, že kromě nevěry není žádného smrtelného hříchu, nebo
že se žádným jiným, jakkoli těžkým a~závažným hříchem než nevěrou neztrácí jednou
přijatá milost, anathema sit.

\canon
Kánon 28: Kdyby někdo řekl, že se hříchem ztrácí milost stejně jako víra, nebo že
víra, která zůstává, byť už není živá, není pravá víra, nebo že ten, který má víru
bez lásky, není křesťan, anathema sit.

\canon
Kánon 29: Kdyby někdo řekl, že ten, kdo po křtu upadl do hříchu, nemůže Boží milostí
opět vstát, nebo že může znovuzískat ztracenou spravedlnost, ale jen pouhou vírou
bez svátosti pokání, jak to svatá římská a~všeobecná církev , poučená Kristem Pánem
a~jeho apoštoly, až do současnosti vyznávala, zachovávala a~učila, anathema sit.

\canon
Kánon 30: Kdyby někdo řekl, že se po přijetí ospravedlnění každému kajícímu hříšníkovi
odpouští vina a~odstraňuje se postižitelnost věčným trestem takovým způsobem, že
nezůstává nic, co by bylo postižitelné časným trestem, který je třeba odpykat buď
v~přítomném životě, nebo později v~očistci, dříve než se bude moci otevřít vstup
do království nebeského, anathema sit.

\canon
Kánon 31: Kdyby někdo řekl, že ospravedlněný hřeší, když jedná dobře se zřetelem
na věčnou odměnu, anathema sit.

\canon
Kánon 32: Kdyby někdo řekl, že dobré skutky ospravedlněného člověka jsou Božími dary
v~tom smyslu, že nejsou i~dobrými zásluhami samotného ospravedlněného, nebo že sám
ospravedlněný dobrými skutky, které koná skrze Boží milost a~zásluhu Ježíše Krista,
jehož je živým údem, ve skutečnosti nezasluhuje rozhojnění milosti, život věčný,
a~— zemře\=li v~milosti~— dosažení věčného života a~rozhojnění milosti, anathema sit.

\canon
Kánon 33: Kdyby někdo řekl, že touto katolickou naukou o~ospravedlnění, vyloženou
posvátným koncilem v~tomto dekretu, se nějakým způsobem umenšuje Boží sláva nebo
zásluhy Ježíše Krista, našeho Pána a~ne že je spíše pravda naší víry i~sláva Boha
a~Krista Ježíše objasněna, anathema sit.

 

 
\autor Překlad: Tomáš Machula

\noindent
\source http://machula.bigbloger.lidovky.cz/c/274989/Tridentsky-koncil-6-cast-1-2.html

\bye

